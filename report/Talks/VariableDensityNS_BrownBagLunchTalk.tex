\documentclass{beamer}
\usepackage{beamerinnerthemecircles, beamerouterthemeshadow}
\usepackage{amsmath, amsfonts, amscd, epsfig, amssymb, amsthm}





%% Create shortcut commands for various fonts and common symbols

%boldface in math mode
\newcommand{\bm}[1]{\mbox{{\boldmath ${#1}$}}}
\newcommand{\C}{\mathbb{C}}
\newcommand{\Du}{\underline{D}}
\newcommand{\del}{\nabla }
\newcommand{\deld}{\nabla \cdot}
\newcommand{\veps}{\varepsilon}
\newcommand{\eps}{\epsilon}
\newcommand{\f}{\textbf{f}}
\newcommand{\fb}{\textbf{f}}
\newcommand{\F}{\mathbb{F}}
\newcommand{\Fb}{\textbf{F}}
\newcommand{\gb}{\textbf{g}}
\newcommand{\grad}{\nabla}
\newcommand{\h}{\textbf{h}}
\newcommand{\kb}{\textbf{k}}
\newcommand{\lap}{\Delta}
\newcommand{\M}{\mathcal{M}}
\newcommand{\N}{\mathbb{N}}
\newcommand{\Norm}{\textbf{N}}
\newcommand{\n}{\textbf{n}}
\newcommand{\vp}{\varphi}
\newcommand{\vph}{\hat{\varphi}}
\newcommand{\p}{\phi}
% note:  \P is already defined to be the paragraph symbol
\newcommand{\Proj}{\mathbb{P}}
\newcommand{\Pcal}{\mathcal{P}}
\newcommand{\Q}{\mathbb{Q}}
\newcommand{\R}{\mathbb{R}}
\newcommand{\rb}{\textbf{r}}
\newcommand{\s}[1]{\mathcal{#1}}
\newcommand{\supp}{\text{supp}}
\newcommand{\Surf}{\textbf{S}}
\newcommand{\tpsi}{\tilde{\psi}}
\newcommand{\ub}{\textbf{u}}
\newcommand{\ut}{\tilde{\textbf{u}}}
\newcommand{\U}{\textbf{U}}
\newcommand{\vb}{\textbf{v}}
\newcommand{\V}{\mathbb{V}}
\newcommand{\wb}{\textbf{w}}
\newcommand{\x}{\textbf{x}}
\newcommand{\xh}{\hat{x}}
\newcommand{\X}{\textbf{X}}
\newcommand{\y}{\textbf{y}}
\newcommand{\yh}{\hat{y}}
\newcommand{\Y}{\textbf{Y}}
\newcommand{\Z}{\mathbb{Z}}


% vectors and tensors
\renewcommand{\vec}[1]{{\bf #1}}
\newcommand{\gvec}[1]{\mbox{{\boldmath ${#1}$}}}
\newcommand{\ten}[1]{\bar{\bm{#1}}}
%derivatives
\newcommand{\od}[2]{\frac{d {#1}}{d {#2}}}
\newcommand{\ods}[2]{\frac{d^2{#1}}{d {{#2}^2}}}
\newcommand{\pd}[2]{\frac{\partial {#1}}{\partial {#2}}}
\newcommand{\pds}[2]{\frac{\partial^2{#1}}{\partial {{#2}^2}}}
\newcommand{\pdsm}[3]{\frac{\partial^2{#1}}{\partial {#2}\,\partial {#3}}}
%funtional analysis
\newcommand{\iprod}[2]{\left( #1, #2 \right)}
\newcommand{\dprod}[2]{\left\langle #1, #2 \right\rangle}
% %real numbers
% \newcommand{\field}[1]{\mathbb{#1}}
% \newcommand{\R}{\field{R}}


%% Declare custom math operators
\DeclareMathOperator{\sech}{sech}
\DeclareMathOperator{\atanh}{atanh}
\DeclareMathOperator{\sign}{sign}
\DeclareMathOperator{\tr}{Trace}
\DeclareMathOperator{\gradsymm}{\nabla_{s}}
\DeclareMathOperator{\divergence}{div}
\DeclareMathOperator{\diag}{diag}
\DeclareMathOperator*{\argmin}{argmin}
\DeclareMathOperator*{\argmax}{argmax}
\DeclareMathOperator{\Span}{Span}
\DeclareMathOperator{\rank}{rank}


%% Sets and systems
\newcommand{\br}[1]{\left\langle #1 \right\rangle}
\newcommand{\paren}[1]{\left(#1\right)}
\newcommand{\sq}[1]{\left[#1\right]}
\newcommand{\set}[1]{\left\{\: #1 \:\right\}}
\newcommand{\setp}[2]{\left\{\, #1\: \middle|\: #2 \, \right\}}
\newcommand{\abs}[1]{\left| #1 \right|}
\newcommand{\norm}[1]{\left\| #1 \right\|}
\newcommand{\system}[1]{\left\{ \begin{array}{rl} #1 \end{array} \right.}

\newcommand{\pf}[2]{\frac{\partial #1}{\partial #2}}
\newcommand{\ipt}[2]{\langle #1,#2 \rangle}
\newcommand{\ip}{\int_{-\infty}^{+\infty}}

\renewcommand{\ker}[1]{\mathcal{N}(#1)}
\newcommand{\ran}[1]{\mathcal{R}(#1)}


%.....variable of integration 
\newcommand{\dx}{\, \mathrm{d}x} 
\newcommand{\dy}{\, \mathrm{d}y} 
\newcommand{\dz}{\, \mathrm{d}z} 
\newcommand{\dA}{\, \mathrm{d}A} 
\newcommand{\da}{\, \mathrm{d}a} 
\newcommand{\dV}{\, \mathrm{d}V} 
\newcommand{\dv}{\, \mathrm{d}v} 
\newcommand{\dt}{\, \mathrm{d}t} 
\newcommand{\ds}{\, \mathrm{d}s}
\newcommand{\dtau}{\, \mathrm{d}\tau}
\newcommand{\Th}{\mathcal{T}_\Delta}
\newcommand{\wt}{\tilde{w}}
\newcommand{\Wt}{\tilde{W}}
%delimiters
\newcommand{\pl}{\left(}
\newcommand{\pr}{\right)}
\newcommand{\sbl}{\left[}
\newcommand{\sbr}{\right]}
\newcommand{\dbl}{\left[\hspace{-0.05cm}\left[}
\newcommand{\dbr}{\right]\hspace{-0.05cm}\right]}
\newcommand{\cbl}{\left\{ }
\newcommand{\cbr}{\right\} }
\newcommand{\eqn}[1]{Equation \ref {eq:#1}}%mwf capitalized for chetn edits 12/11/07 
\newcommand{\Eqn}[1]{Equation \ref {eq:#1}} 
\newcommand{\eqnst}[2]{equations \ref{eq:#1} and \ref{eq:#2}} 
\newcommand{\Eqnst}[2]{Equations \ref{eq:#1} and \ref{eq:#2}} 
\newcommand{\eqns}[2]{equations \ref{eq:#1}--\ref{eq:#2}} 
\newcommand{\Eqns}[2]{Equations \ref{eq:#1}--\ref{eq:#2}}
\newcommand{\msection}[1]{ \vspace{.2in} {\noindent \bf #1}.}
\newcommand{\for}{\mbox{for}\quad}
% \newcommand{\argmin}{\mbox{argmin}}
% \newcommand{\argmax}{\mbox{argmax}}
\newcommand{\fig}[1]{figure \ref{fig:#1}} 
\newcommand{\Fig}[1]{Figure \ref{fig:#1}} 
\newcommand{\figst}[2]{figures \ref {fig:#1} and \ref {fig:#2}} 
\newcommand{\Figst}[2]{Figures \ref {fig:#1} and \ref {fig:#2}} 
\newcommand{\figs}[2]{figures \ref{fig:#1}--\ref{fig:#2}} 
\newcommand{\Figs}[2]{Figures \ref{fig:#1}--\ref{fig:#2}}
\newcommand{\tab}[1]{table \ref {tab:#1}} 
\newcommand{\Tab}[1]{Table \ref {tab:#1}} 
\newcommand{\tabst}[2]{tables \ref {tab:#1} and \ref {tab:#2}} 
\newcommand{\Tabst}[2]{Tables \ref {tab:#1} and \ref {tab:#2}} 
\newcommand{\tabs}[2]{tables \ref{tab:#1}--\ref{tab:#2}} 
\newcommand{\Tabs}[2]{Tables \ref{tab:#1}--\ref{tab:#2}}
%%velocity                                                                                                                                                                                                            
\newcommand{\vel}{\gvec{\sigma}}
\newenvironment{neqnarray}[1]{\begin{minipage}[t]{6.5in}  \begin{minipage}[b]{1.0in} #1 \end{minipage}  \begin{minipage}[b]{5.5in}\begin{eqnarray}}{\end{eqnarray}\end{minipage}\end{minipage}}
\newcommand{\bneqnarray}[2]{\\ \\ \fbox{\begin{neqnarray}{#1} #2 \end{neqnarray}}\\ \\ \noindent}

%element
\newcommand{\elem}{\Omega}
%element boundary (ind. of element)
%\newcommand{\face}{\partial \Omega}
\newcommand{\face}{\gamma}
%mesh nodes
\newcommand{\node}{\vec x}
%node star
\newcommand{\nodestar}[1]{\mathcal{E}({#1})}
%faces in element not on Neumann boundary
\newcommand{\dirIntFaces}[1]{\mathcal{F}_{i,d}({#1})}
%faces in element not on physical boundary
\newcommand{\intFaces}[1]{\mathcal{F}_{i}({#1})}
%nodes beloging to an element
\newcommand{\elemnodes}[1]{\mathcal{N}({#1})}
%nodes beloging to an element boundary
\newcommand{\facenodes}[1]{\mathcal{N}({#1})}
%left and right elements at a face
\newcommand{\lelem}{\Omega_{\ell}}
\newcommand{\relem}{\Omega_{r}}
%the left and right identifiers
\newcommand{\eleft}[1]{e_{\ell}({#1})}
\newcommand{\eright}[1]{e_{r}({#1})}
%local numbering on nodestars
\newcommand{\elemstar}{e^{\ast}}
\newcommand{\estarleft}[1]{e^{\ast}_{\ell}({#1})}
\newcommand{\estarright}[1]{e^{\ast}_{r}({#1})}
%left and right normals to face
\newcommand{\lnormal}{\vec{n}_{\ell}}
\newcommand{\rnormal}{\vec{n}_{r}}
%unique normal on face
\newcommand{\fnormal}{\vec{n}_{f}}
%local indeces on left and right
\newcommand{\ileft}{i_{\ell}}
\newcommand{\iright}{i_{r}}
%jump operator
%mwf orig
%\newcommand{\jump}[1]{\dbl #1 \dbr}
%\newcommand{\jump}[2][-0.075cm]{\left[\hspace{#1} \left[ #2 \right]\hspace{#1} \right]}
\newcommand{\jump}[2][\!]{\left[ #1 \left[ #2 \right]#1 \right]}
%multiscale formalism
\newcommand{\strongRes}{\mathcal{R}}
\newcommand{\Lop}{\mathcal{L}}
\newcommand{\LopStar}{\mathcal{L}^{\ast}}
\newcommand{\Ls}{\mathcal{L}_s}
\newcommand{\LsStar}{\mathcal{L}_s^{\ast}}
\newcommand{\LsStarApprox}{\mathcal{L}^{\ast}_{s,h}}
\newcommand{\LsHat}{\hat{\mathcal{L}}_s}
%richards equation stuff
\newcommand{\psk}{$p$-$s$-$k$}

%tables and display convenience
\newcommand{\tx}[1]{\times 10^{#1}}
%... any local macros necessary
%element identifier
\newcommand{\E}{\mathcal{E}}
%ref. element identifier
\newcommand{\hE}{\hat{\E}}
%ref. edge identifier
\newcommand{\he}{\hat{e}}
%triangulation
\newcommand{\Mh}{\mathcal{M}^h}
%shape function on element boundary
\newcommand{\eN}{\tilde{N}}
%d-1 reference element shape function
\newcommand{\heN}{\widehat{\tilde{N}}}
%for matrices
\newfont {\matFont}{cmssbx10 at 12 pt} 
\newcommand{\mat}[1]{\hbox  {\matFont #1}}
%reference element integration d's
\newcommand{\dxh}{\, \mathrm{d}\hat{x}\mathrm{d}\hat{y}}
\newcommand{\dsh}{\, \mathrm{d}\hat{s}}
\newcommand{\dth}{\, \mathrm{d}\hat{t}}
\newcommand{\ptab}{\hspace*{12pt}}

% Space functions
\newcommand{\wDel}{w_\Delta}
\newcommand{\delw}{\delta w}
\newcommand{\vDel}{v_\Delta}
\newcommand{\delv}{\delta v}
\newcommand{\uDel}{u_\Delta}
\newcommand{\delu}{\delta u}







\mode<presentation> {
    \usetheme{Frankfurt} %Bergen, Berkely, Berlin, Boadilla, CambridgeUS, Darmstadt,
                          %Frankfurt, Goettingen, Singapore, Warsaw
    \usecolortheme{default} %beetle, seahorse, wolverine, dolphin, beaver
}

\title[Navier-Stokes]{Variable Density Incompressible Navier-Stokes Fractional Time-Stepping Algorithms}

\author[Patty]{Spencer Patty}

\institute[TAMU Math]{Hydrologic Systems Branch, CHL, ERDC\\Department of Mathematics, Texas A\&M
University }

\date[June 2015]{June 25, 2015}

\usefonttheme[onlymath]{serif}


%\pgfdeclareimage[height=1cm]{university-logo}{byu_logo}
%\logo{\pgfuseimage{university-logo}}

% If you want to display your table of contents at every Section/Subsection:
%\AtBeginSection[] {
%\begin{frame}<beamer>{Table of Contents}
%\tableofcontents[currentsection,currentsubsection]
%\end{frame}
%}

\begin{document}

\maketitle

\frame{\tableofcontents}

\section[Intro]{Introduction to the Navier-Stokes Equations}

% 1
\begin{frame}
  \frametitle{Why Navier-Stokes?}
  \begin{block}{Motion of Fluids}
      The Navier-Stokes equations are a set of nonlinear partial differential equations that model flow of liquids or gases.
    \end{block}
      \begin{columns}
          \begin{column}{5cm} % width = 5cm
              \begin{itemize}
                  \item<2-> Weather modelling
                  \item<2-> Air currents in atmosphere
                  \item<2-> Flow or waves in rivers, oceans, shallow water, etc.
                  \item<2-> Fire/ Smoke propation and flow
                  \item<2-> Aircraft design
              \end{itemize}
          \end{column}
          \begin{column}{5cm} % width = 5cm
              \begin{itemize}
                  \item<2-> Low Mach number gases flow
                  \item<2-> Water flow in a pipe
                  \item<2-> Blood flow in circulatory systems
                  \item<2-> Movie animations
                  \item<2-> Air/Water interactions
                  \item<2-> Cell motility (ex: white blood cell in liquid matrix)
              \end{itemize}
          \end{column}
      \end{columns}
\end{frame}


%2
\begin{frame}
  \frametitle{Navier-Stokes Equations (sans Energy Equations)}
  \begin{itemize}
  \item<1->\begin{block}{Conservation of Mass}
    \begin{equation*}
      \rho_t + \nabla\cdot\left( \rho \ub \right) = 0
    \end{equation*}
  \end{block}
  We can expand the divergence and regroup to use \textbf{material derivative} $\left(\frac{D}{Dt} = \frac{\partial}{\partial t} + \ub\cdot\nabla\right)$.  Thus,
  \begin{equation*}
    \frac{D \rho}{D t} + \rho\nabla\cdot \ub = 0
  \end{equation*}
  \item<2->\begin{block}{Conservation of Momentum}
    \begin{equation*}
      \frac{\partial}{\partial t}\left( \rho \ub \right) + \nabla\cdot\left( \rho \ub\otimes\ub \right) = \nabla\cdot\left( \tau - p\mathbb{I}\right) + \rho \gb + \fb
    \end{equation*}
  \end{block}
    $\tau$ is the \textbf{stress tensor} which can be quite complicated.  However, often we can use $\tau = \mu \left(\nabla \ub + \nabla\ub^{T}\right) = 2\mu\nabla^S\ub$ or more simplified $\tau = \mu\nabla\ub$. 
    
  \end{itemize}
\end{frame}

%3
\begin{frame}
  \frametitle{Incompressible Navier-Stokes}
  \begin{itemize}
    \item<1->
  \begin{block}{Compressibility}
    Compressible fluid = changes in pressure or temperature results in changes in density.
    \begin{equation*}
      \mbox{incompressible fluid} \: \Leftrightarrow \: \frac{D\rho}{Dt} = 0 \: \Leftrightarrow \: \nabla\cdot \ub = 0
    \end{equation*}
  \end{block}
  Thus we often say $\nabla\cdot\ub=0$ is the incompressibility constraint.
  \item<2-> Recalling $\ub\otimes\ub = \ub\ub^{T}$, $\nabla\cdot(p\mathbb{I}) = \nabla p$, and using $\tau = \mu\nabla\ub$, the previous equations reduce to
  \begin{block}{Non-Conservative Incompressible Navier-Stokes}
   \begin{equation*}
     \begin{cases}
       \rho_t + \nabla\cdot\left(\rho \ub \right) = 0 &\\
       \rho\left( \ub_t + \ub\cdot\nabla\ub \right) + \nabla p - \nabla\cdot\left( \mu\nabla\ub \right) = \rho\gb + \fb & \\
       \nabla\cdot \ub = 0 &
     \end{cases}
   \end{equation*}
  \end{block}
  \end{itemize}
\end{frame}


%
%
%
\section[Projection]{Constant Density Projection Schemes}

%4
\begin{frame}
  \frametitle{Chorin-Temam Pressure Correction Algorithm (constant $\rho$)}
  Compute $(\ut^{k+1}, \ub^{k+1}, p^{k+1})$ by solving two steps
  \begin{block}{}
    \begin{align*}
    &\rho\left( \frac{\ut^{k+1} - \ub^{k}}{\tau}  + \ut^{k}\cdot\nabla\ut^{k+1}\right) - \mu\Delta \ut^{k+1} = \fb^{k+1},  \hspace{0.5cm} \left.\ut^{k+1}\right|_{\partial\Omega}=0\\
    &\begin{cases}
      \frac{\rho}{\tau}\left( \ub^{k+1} - \ut^{k+1}\right) + \nabla p^{k+1}= 0\\
      \nabla\cdot \ub^{k+1} = 0,\hspace{1cm}  \left.\ub^{k+1}\cdot\n\right|_{\partial\Omega} = 0 
    \end{cases}
    \end{align*}
  \end{block}
  First viscous effects,  then incompressibility.  Notice that the second equation gives $\ut^{k+1}$ in terms of a divergence free part and a gradient part (curl free)
  \begin{equation}
    \ut^{k+1} = \ub^{k+1} + \frac{\tau}{\rho} \nabla p^{k+1}
  \end{equation}
  so $\ub^{k+1} = P_H \ut^{k+1}$ is the projection into the divergence free space $H=\{ \vb\in L^2 \:|\: \nabla\cdot\vb = 0,  \left.\vb\cdot\n\right|_{\partial\Omega}=0 \}$.
  
\end{frame}


%5
\begin{frame}
  \frametitle{Incremental Pressure Correction Algorithm (constant $\rho$)}
  Compute $(\ut^{k+1}, \ub^{k+1}, p^{k+1})$ by solving two steps
  \begin{block}{}
    \begin{align*}
    &\rho\left( \frac{\ut^{k+1} - \ub^{k}}{\tau}  + \ut^{k}\cdot\nabla\ut^{k+1}\right) - \mu\Delta \ut^{k+1} + \nabla p^{k} = \fb^{k+1},  \hspace{0.1cm} \left.\ut^{k+1}\right|_{\partial\Omega}=0\\
    &\begin{cases}
      \frac{\rho}{\tau}\left( \ub^{k+1} - \ut^{k+1}\right) + \nabla \left(p^{k+1}-p^{k}\right)= 0\\
      \nabla\cdot \ub^{k+1} = 0,\hspace{1cm}  \left.\ub^{k+1}\cdot\n\right|_{\partial\Omega} = 0 
    \end{cases}
    \end{align*}
  \end{block}
  This does a little better job than before.  We still have projection property $\ub^{k+1} = P_H \ut^{k+1}$.  Notice that the second equation implies \[\left.\nabla p^{k+1}\cdot\n\right|_{\partial\Omega} = \left.\nabla p^{k}\cdot\n\right|_{\partial\Omega} = \dots \left.\nabla p^{0}\cdot\n\right|_{\partial\Omega},\]
  an artificial neumann bc that limits accuracy of pressure.
\end{frame}

%
%
%
\section[Penalty]{Penalty Methods Point of View for Variable Density}

%6
\begin{frame}
  \frametitle{Perturbed system (constant $\rho$)}
  It is possible to solve for incremental pressure-correction algorithm in terms of only the non-solenoidal velocity $\tilde{\ub}^{k}$ and pressure $p^{k}$ in the form
  \begin{equation*}
    \begin{cases}
      \rho\left(\frac{\tilde{\ub}^{k+1} - \tilde{\ub}^{k}}{\tau} + \tilde{\ub}^{k}\cdot\nabla\tilde{\ub}^{k+1} \right) - \mu\Delta\tilde{\ub}^{k+1} + \nabla\left(p^{k} +\phi^{k}  \right) = \fb^{k+1}\\
      \nabla\cdot\tilde{\ub}^{k+1} - \frac{\tau}{\rho}\Delta\phi^{n+1} = 0, \hspace{1cm}\left.\partial_{\n} \phi\right|_{\partial\Omega} = 0\\
      p^{k+1} = p^{k} + \phi^{k+1}
    \end{cases}
  \end{equation*}
  This can be seen as a discrete version of the following system % (formally $\mathcal{O}(\veps^2)$) perturbation of constant density navier-stokes equation
  \begin{block}{}
  \begin{equation*}
    \begin{cases}
      \rho\left(\ub_t + \ub\cdot\nabla\ub  \right) + \nabla p -\mu\Delta\ub = \fb, & \left.\ub\right|_{\partial\Omega} = 0\\
      \nabla\cdot\ub - \frac{\veps}{\rho}\Delta \phi = 0, & \left.\partial_{\n} \phi\right|_{\partial\Omega} = 0\\
      \veps p_t = \phi
    \end{cases}
    \end{equation*}
  \end{block}
  where we have replaced difference quotients with time derivatives and substituted $\veps = \tau$ for the remaining $\tau$'s.
\end{frame}

%
%
%
\section[Variable Density]{Variable Density BDF1 Algorithm}

%7
\begin{frame}
  \frametitle{BDF1 Rotational Incremental Algorithm}
  \begin{itemize}
    \item<1-> We use penalty perspective to deal with variable density and modify algoirthms accordingly.
    \item <1-> Given 
    \[(\rho^{k}, \ub^{k}, p^{k})\in \Proj^2\times\left(\Proj^2\right)^d\times\Proj^1,\]  
    we solve progressively for 
    \begin{equation*}
      (\rho^{k+1}, \ub^{k+1}, p^{k+1})\in \Proj^2\times\left(\Proj^2\right)^d\times\Proj^1,
    \end{equation*}
    one at a time using all the currently available information. 
    
    There are three stages to the algorithm: 
    \begin{enumerate}
      \item Density update
      \item Velocity update
      \item Pressure update.
    \end{enumerate}
     
    \item<2-> This is called a \textbf{split} operator algorithm since we have split the fully coupled interdependence into a sequential set of operations with only backward dependence.
    \end{itemize}
\end{frame}

%8
\begin{frame}
  \frametitle{BDF1 Density Update:}
  \begin{equation*}\boxed{
    \rho_t + \nabla\cdot\left(\rho\ub\right) = 0}
  \end{equation*}
  \vfill
  Find $\rho_h^{k+1}\in \Proj^2$ that satisfies the weak form of the following equation
  \begin{block}{}
    \begin{equation*}
      \frac{\rho^{k+1} - \rho^k}{\tau} + \nabla\cdot\left(\rho^{k+1}\ub^{k} \right) - \frac{\rho^{k+1}}{2}\nabla\cdot\ub^{k} = 0.
    \end{equation*}
  \end{block}
  Use any maximum preserving hyperbolic solver to do so.
\end{frame}
      
%9
 \begin{frame}
   \frametitle{BDF1 Velocity Update:}
    \begin{equation*}\boxed{
      \rho\left(\ub_t + \ub\cdot\nabla\ub\right) + \nabla p - \nabla\left(\mu \nabla\ub\right) = f}
    \end{equation*}
    \vfill
    Find $\ub^{k+1}_h\in\left(\Proj^2\right)^d$ such that for all $\vb_h \in \left(\Proj^2\right)^d$ we have,
    \begin{block}{}
    \begin{equation*}
      \begin{split}
        \dprod{\rho_h^{k} \frac{\ub_h^{k+1}-\ub_h^{k} }{\tau} }{\vb_h} + \dprod{ \rho_h^{k+1}\ub_h^{k}\cdot\nabla\ub_h^{k+1} }{\vb_h} \\
         + \frac{1}{2}\dprod{ \left(\frac{\rho_h^{k+1} - \rho_h^{k}}{\tau} + \nabla\cdot\left(\rho_h^{k+1}\ub_h^{k}\right)  \right )\ub_h^{k+1} }{\vb_h} &\\
         +\dprod{\nabla \left(2p_h^{k} - p_h^{k-1}\right)}{\vb_h}+ \mu\dprod{ \nabla \ub_h^{k+1} }{\nabla\vb_h} &= \dprod{\fb^{k+1}}{\vb_h}.
      \end{split}
    \end{equation*}
    \end{block}
  \end{frame}
  
  %10
  \begin{frame}
    \frametitle{BDF1 Pressure Update: (with Rotational Term)}
    \begin{equation*}\boxed{
      \begin{cases}
      -\Delta \phi = -\frac{\chi}{\tau}\nabla\cdot\ub,\hspace{0.5cm} \left.\partial_{\n}\phi\right|_{\partial\Omega} = 0\\
      p^{k+1} = p^{k} + \phi - \mu\nabla\cdot\ub
      \end{cases}}
    \end{equation*}
    \vfill
    Find the pressure correction $\phi_h^{k+1}\in \Proj^1$ such that for all $r_h\in \Proj^1$,
    \begin{block}{}
    \begin{equation*}
      \dprod{\nabla\phi_h^{k+1}}{\nabla r_h} = \frac{\chi}{\tau}\dprod{\ub_h^{k+1}}{\nabla r_h}
    \end{equation*}
    \end{block}
    where $\chi = \displaystyle\min_{\x\in\overline{\Omega}}\rho_0(\x)$.\newline

    Find the pressure $p_h^{k+1}\in \Proj^1$ such that for all $r_h\in \Proj^1$
    \begin{block}{}
    \begin{equation*}
      \dprod{p_h^{k+1}}{r_h} = \dprod{\phi_h^{k+1} + p_h^{k}}{r_h} + \mu\dprod{\ub_h^{k+1}}{\nabla r_h}.
    \end{equation*}
    \end{block}
  \end{frame}

%
%
%
\section[Convergence]{Convergence Rates}

%11
\begin{frame}
  \frametitle{Convergence Rates Incremental BDF1} 
  Let $(\ub_h^k, p_h^{k})\in \left(\mathbb{P}^2\right)^{d}\times\mathbb{P}^1$.  \newline
  
  Suppose initial pressure is uniformly bounded in $H^1$ and the initial approximation of velocity is discretely divergence free,
  \begin{equation*}
    \|p_h^0\|_{H^1(\Omega)} \leq c \hspace{0.5cm} \mbox{and} \hspace{0.5cm} \dprod{\ub_h^0}{\nabla r_h} = 0 \hspace{0.25cm} \forall r_h\in \Proj^1
  \end{equation*}
    Then, Guermond and Salgado (2011) proved
  \begin{block}{Theorem (Incremental Non Rotational BDF1 Algorithm)}
    \begin{align*}
      \left\|\left(\ub\right)_{\tau} - \left(\ub_h\right)_{\tau}\right\|_{\ell^{\infty}\left(L^2(\Omega)\right)} &\leq c\left( \tau + h^2\right)\\
      \left\|\left(\ub\right)_{\tau} - \left(\ub_h\right)_{\tau}\right\|_{\ell^{2}\left(H^1(\Omega)\right)} &\leq c\left(\tau + h \right)\\
      \left\|\left(p\right)_{\tau} - \left(p_h\right)_{\tau}\right\|_{\ell^{2}\left(L^2(\Omega)\right)} &\leq c\left(\tau + h \right)
    \end{align*}
  \end{block}
  
\end{frame}

%12
\begin{frame}
  \frametitle{Convergence Rates Incremental Rotational BDF2} 
  Under the same properties above, by replacing time derivatives with second order approximation and using
  \begin{equation*}
    \ub^{*}_h = \ub_h^{k} + \frac{\tau^{k+1}}{\tau^{k}}\left( \ub_h^{k} - \ub_h^{k-1} \right)
  \end{equation*}
  for advection term, we conjecture
  \begin{block}{Conjecture (Incremental Rotational BDF2 Algorithm)}
    \begin{align*}
      \left\|\left(\ub\right)_{\tau} - \left(\ub_h\right)_{\tau}\right\|_{\ell^{\infty}\left(L^2(\Omega)\right)} &\leq c\left( \tau^2 + h^2\right)\\
      \left\|\left(\ub\right)_{\tau} - \left(\ub_h\right)_{\tau}\right\|_{\ell^{2}\left(H^1(\Omega)\right)} &\leq c\left(\tau^{3/2} + h \right)\\
      \left\|\left(p\right)_{\tau} - \left(p_h\right)_{\tau}\right\|_{\ell^{2}\left(L^2(\Omega)\right)} &\leq c\left(\tau^{3/2} + h \right)
    \end{align*}
  \end{block}
  
\end{frame}

%13
\begin{frame}[c]
    \begin{center}
      \huge Questions?
    \end{center}
\end{frame}

%
% \begin{frame}
%     \frametitle{Overview of Numerical Methods}
%     \begin{itemize}
%         \item<1->Finite Difference
%         \item<2-> Finite Element
%         \item<3->Level Set Methods
%     \end{itemize}
% \end{frame}
%
% \begin{frame}
%     \frametitle{What is a Finite Difference?}
%     \begin{columns}
%         \begin{column}{5cm} % width = 5cm
%             \begin{itemize}
%                 \item Hello
%                 \item World
%             \end{itemize}
%         \end{column}
%         \begin{column}{5cm} % width = 5cm
%             \begin{itemize}
%                 \item Hola
%                 \item Mundo
%             \end{itemize}
%         \end{column}
%     \end{columns}
% \end{frame}
%
% \begin{frame}
%     \frametitle{This is content}
%     \begin{itemize}
%         \item<1,2>Test 1
%         \item<2>Test 2
%         \item<3>Test 3
%     \end{itemize}
% \end{frame}
%
% \begin{frame}
%     \frametitle{This is content subsection}
%     \begin{block}{Blocktitle}
%         This is a block.
%     \end{block}
%     \begin{theorem}
%         If $f:[a,b] \to \R$ is continuous then $f$ attains a
%         minimum and maximum value on $[a,b]$.
%     \end{theorem}
% \end{frame}

\end{document}
