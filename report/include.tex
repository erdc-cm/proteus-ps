%%%%%%%%%%%%%%%%%%%%%%%%%%%%%%%%%%%%%%%%%%%%%%%%%%%%%%%%%%%%%%%%%%%%%%%%
%begin macros section from qfrkdg-macros
%%%%%%%%%%%%%%%%%%%%%%%%%%%%%%%%%%%%%%%%%%%%%%%%%%%%%%%%%%%%%%%%%%%%%%%%

\theoremstyle{plain}
\newtheorem{theorem}{Theorem}
% \newtheorem{theorem}{Theorem}[section]
\newtheorem{proposition}[theorem]{Proposition}
\newtheorem{corollary}[theorem]{Corollary}
\newtheorem{lemma}[theorem]{Lemma}

\theoremstyle{definition}
\newtheorem{definition}[theorem]{Definition}
\newtheorem{example}[theorem]{Example}

\theoremstyle{remark}
\newtheorem*{remark}{Remark}
\newtheorem*{claim}{Claim}




%% Create shortcut commands for various fonts and common symbols

%boldface in math mode
\newcommand{\Ab}{\mathbf{A}}
\newcommand{\bm}[1]{\mbox{{\boldmath ${#1}$}}}
\newcommand{\C}{\mathbb{C}}
\newcommand{\Du}{\underline{D}}
\newcommand{\del}{\nabla }
\newcommand{\deld}{\nabla \cdot}
\newcommand{\veps}{\varepsilon}
\newcommand{\eps}{\epsilon}
\newcommand{\f}{\textbf{f}}
\newcommand{\fb}{\textbf{f}}
\newcommand{\F}{\mathbb{F}}
\newcommand{\Fb}{\textbf{F}}
\newcommand{\gb}{\textbf{g}}
\newcommand{\grad}{\nabla}
\newcommand{\h}{\textbf{h}}
\newcommand{\kb}{\textbf{k}}
\newcommand{\lap}{\Delta}
\newcommand{\M}{\mathcal{M}}
\newcommand{\N}{\mathbb{N}}
\newcommand{\Norm}{\textbf{N}}
\newcommand{\n}{\textbf{n}}
\newcommand{\vp}{\varphi}
\newcommand{\vph}{\hat{\varphi}}
\newcommand{\p}{\phi}
% note:  \P is already defined to be the paragraph symbol
\newcommand{\Proj}{\mathbb{P}}
\newcommand{\Pcal}{\mathcal{P}}
\newcommand{\Q}{\mathbb{Q}}
\newcommand{\R}{\mathbb{R}}
\newcommand{\rb}{\textbf{r}}
\newcommand{\s}[1]{\mathcal{#1}}
\newcommand{\supp}{\text{supp}}
\newcommand{\Surf}{\textbf{S}}
\newcommand{\tpsi}{\tilde{\psi}}
\newcommand{\ub}{\textbf{u}}
\newcommand{\ut}{\mathbf{\tilde{u}}}
\newcommand{\U}{\textbf{U}}
\newcommand{\vb}{\textbf{v}}
\newcommand{\V}{\mathbb{V}}
\newcommand{\wb}{\textbf{w}}
\newcommand{\x}{\textbf{x}}
\newcommand{\xh}{\hat{x}}
\newcommand{\X}{\textbf{X}}
\newcommand{\y}{\textbf{y}}
\newcommand{\yh}{\hat{y}}
\newcommand{\Y}{\textbf{Y}}
\newcommand{\Z}{\mathbb{Z}}


% vectors and tensors
\renewcommand{\vec}[1]{{\bf #1}}
\newcommand{\gvec}[1]{\mbox{{\boldmath ${#1}$}}}
\newcommand{\ten}[1]{\bar{\bm{#1}}}
%derivatives
\newcommand{\od}[2]{\frac{d {#1}}{d {#2}}}
\newcommand{\ods}[2]{\frac{d^2{#1}}{d {{#2}^2}}}
\newcommand{\pd}[2]{\frac{\partial {#1}}{\partial {#2}}}
\newcommand{\pds}[2]{\frac{\partial^2{#1}}{\partial {{#2}^2}}}
\newcommand{\pdsm}[3]{\frac{\partial^2{#1}}{\partial {#2}\,\partial {#3}}}
%funtional analysis
\newcommand{\iprod}[2]{\left( #1, #2 \right)}
\newcommand{\dprod}[2]{\left\langle #1, #2 \right\rangle}
% %real numbers
% \newcommand{\field}[1]{\mathbb{#1}}
% \newcommand{\R}{\field{R}}


%% Declare custom math operators
\DeclareMathOperator{\sech}{sech}
\DeclareMathOperator{\atanh}{atanh}
\DeclareMathOperator{\sign}{sign}
\DeclareMathOperator{\tr}{Trace}
\DeclareMathOperator{\gradsymm}{\nabla_{s}}
\DeclareMathOperator{\divergence}{div}
\DeclareMathOperator{\diag}{diag}
\DeclareMathOperator*{\argmin}{argmin}
\DeclareMathOperator*{\argmax}{argmax}
\DeclareMathOperator{\Span}{Span}
\DeclareMathOperator{\rank}{rank}


%% Sets and systems
\newcommand{\br}[1]{\left\langle #1 \right\rangle}
\newcommand{\paren}[1]{\left(#1\right)}
\newcommand{\sq}[1]{\left[#1\right]}
\newcommand{\set}[1]{\left\{\: #1 \:\right\}}
\newcommand{\setp}[2]{\left\{\, #1\: \middle|\: #2 \, \right\}}
\newcommand{\abs}[1]{\left| #1 \right|}
\newcommand{\norm}[1]{\left\| #1 \right\|}
\newcommand{\system}[1]{\left\{ \begin{array}{rl} #1 \end{array} \right.}

\newcommand{\pf}[2]{\frac{\partial #1}{\partial #2}}
\newcommand{\ipt}[2]{\langle #1,#2 \rangle}
\newcommand{\ip}{\int_{-\infty}^{+\infty}}

\renewcommand{\ker}[1]{\mathcal{N}(#1)}
\newcommand{\ran}[1]{\mathcal{R}(#1)}


%.....variable of integration 
\newcommand{\dx}{\, \mathrm{d}x} 
\newcommand{\dy}{\, \mathrm{d}y} 
\newcommand{\dz}{\, \mathrm{d}z} 
\newcommand{\dA}{\, \mathrm{d}A} 
\newcommand{\da}{\, \mathrm{d}a} 
\newcommand{\dV}{\, \mathrm{d}V} 
\newcommand{\dv}{\, \mathrm{d}v} 
\newcommand{\dt}{\, \mathrm{d}t} 
\newcommand{\ds}{\, \mathrm{d}s}
\newcommand{\dtau}{\, \mathrm{d}\tau}
\newcommand{\Th}{\mathcal{T}_\Delta}
\newcommand{\wt}{\tilde{w}}
\newcommand{\Wt}{\tilde{W}}
%delimiters
\newcommand{\pl}{\left(}
\newcommand{\pr}{\right)}
\newcommand{\sbl}{\left[}
\newcommand{\sbr}{\right]}
\newcommand{\dbl}{\left[\hspace{-0.05cm}\left[}
\newcommand{\dbr}{\right]\hspace{-0.05cm}\right]}
\newcommand{\cbl}{\left\{ }
\newcommand{\cbr}{\right\} }
\newcommand{\eqn}[1]{Equation \ref {eq:#1}}%mwf capitalized for chetn edits 12/11/07 
\newcommand{\Eqn}[1]{Equation \ref {eq:#1}} 
\newcommand{\eqnst}[2]{equations \ref{eq:#1} and \ref{eq:#2}} 
\newcommand{\Eqnst}[2]{Equations \ref{eq:#1} and \ref{eq:#2}} 
\newcommand{\eqns}[2]{equations \ref{eq:#1}--\ref{eq:#2}} 
\newcommand{\Eqns}[2]{Equations \ref{eq:#1}--\ref{eq:#2}}
\newcommand{\msection}[1]{ \vspace{.2in} {\noindent \bf #1}.}
\newcommand{\for}{\mbox{for}\quad}
% \newcommand{\argmin}{\mbox{argmin}}
% \newcommand{\argmax}{\mbox{argmax}}
\newcommand{\fig}[1]{figure \ref{fig:#1}} 
\newcommand{\Fig}[1]{Figure \ref{fig:#1}} 
\newcommand{\figst}[2]{figures \ref {fig:#1} and \ref {fig:#2}} 
\newcommand{\Figst}[2]{Figures \ref {fig:#1} and \ref {fig:#2}} 
\newcommand{\figs}[2]{figures \ref{fig:#1}--\ref{fig:#2}} 
\newcommand{\Figs}[2]{Figures \ref{fig:#1}--\ref{fig:#2}}
\newcommand{\tab}[1]{table \ref {tab:#1}} 
\newcommand{\Tab}[1]{Table \ref {tab:#1}} 
\newcommand{\tabst}[2]{tables \ref {tab:#1} and \ref {tab:#2}} 
\newcommand{\Tabst}[2]{Tables \ref {tab:#1} and \ref {tab:#2}} 
\newcommand{\tabs}[2]{tables \ref{tab:#1}--\ref{tab:#2}} 
\newcommand{\Tabs}[2]{Tables \ref{tab:#1}--\ref{tab:#2}}
%%velocity                                                                                                                                                                                                            
\newcommand{\vel}{\gvec{\sigma}}
\newenvironment{neqnarray}[1]{\begin{minipage}[t]{6.5in}  \begin{minipage}[b]{1.0in} #1 \end{minipage}  \begin{minipage}[b]{5.5in}\begin{eqnarray}}{\end{eqnarray}\end{minipage}\end{minipage}}
\newcommand{\bneqnarray}[2]{\\ \\ \fbox{\begin{neqnarray}{#1} #2 \end{neqnarray}}\\ \\ \noindent}

%element
\newcommand{\elem}{\Omega}
%element boundary (ind. of element)
%\newcommand{\face}{\partial \Omega}
\newcommand{\face}{\gamma}
%mesh nodes
\newcommand{\node}{\vec x}
%node star
\newcommand{\nodestar}[1]{\mathcal{E}({#1})}
%faces in element not on Neumann boundary
\newcommand{\dirIntFaces}[1]{\mathcal{F}_{i,d}({#1})}
%faces in element not on physical boundary
\newcommand{\intFaces}[1]{\mathcal{F}_{i}({#1})}
%nodes beloging to an element
\newcommand{\elemnodes}[1]{\mathcal{N}({#1})}
%nodes beloging to an element boundary
\newcommand{\facenodes}[1]{\mathcal{N}({#1})}
%left and right elements at a face
\newcommand{\lelem}{\Omega_{\ell}}
\newcommand{\relem}{\Omega_{r}}
%the left and right identifiers
\newcommand{\eleft}[1]{e_{\ell}({#1})}
\newcommand{\eright}[1]{e_{r}({#1})}
%local numbering on nodestars
\newcommand{\elemstar}{e^{\ast}}
\newcommand{\estarleft}[1]{e^{\ast}_{\ell}({#1})}
\newcommand{\estarright}[1]{e^{\ast}_{r}({#1})}
%left and right normals to face
\newcommand{\lnormal}{\vec{n}_{\ell}}
\newcommand{\rnormal}{\vec{n}_{r}}
%unique normal on face
\newcommand{\fnormal}{\vec{n}_{f}}
%local indeces on left and right
\newcommand{\ileft}{i_{\ell}}
\newcommand{\iright}{i_{r}}
%jump operator
%mwf orig
%\newcommand{\jump}[1]{\dbl #1 \dbr}
%\newcommand{\jump}[2][-0.075cm]{\left[\hspace{#1} \left[ #2 \right]\hspace{#1} \right]}
\newcommand{\jump}[2][\!]{\left[ #1 \left[ #2 \right]#1 \right]}
%multiscale formalism
\newcommand{\strongRes}{\mathcal{R}}
\newcommand{\Lop}{\mathcal{L}}
\newcommand{\LopStar}{\mathcal{L}^{\ast}}
\newcommand{\Ls}{\mathcal{L}_s}
\newcommand{\LsStar}{\mathcal{L}_s^{\ast}}
\newcommand{\LsStarApprox}{\mathcal{L}^{\ast}_{s,h}}
\newcommand{\LsHat}{\hat{\mathcal{L}}_s}
%richards equation stuff
\newcommand{\psk}{$p$-$s$-$k$}

%tables and display convenience
\newcommand{\tx}[1]{\times 10^{#1}}
%... any local macros necessary
%element identifier
\newcommand{\E}{\mathcal{E}}
%ref. element identifier
\newcommand{\hE}{\hat{\E}}
%ref. edge identifier
\newcommand{\he}{\hat{e}}
%triangulation
\newcommand{\Mh}{\mathcal{M}^h}
%shape function on element boundary
\newcommand{\eN}{\tilde{N}}
%d-1 reference element shape function
\newcommand{\heN}{\widehat{\tilde{N}}}
%for matrices
\newfont {\matFont}{cmssbx10 at 12 pt} 
\newcommand{\mat}[1]{\hbox  {\matFont #1}}
%reference element integration d's
\newcommand{\dxh}{\, \mathrm{d}\hat{x}\mathrm{d}\hat{y}}
\newcommand{\dsh}{\, \mathrm{d}\hat{s}}
\newcommand{\dth}{\, \mathrm{d}\hat{t}}
\newcommand{\ptab}{\hspace*{12pt}}

% Space functions
\newcommand{\wDel}{w_\Delta}
\newcommand{\delw}{\delta w}
\newcommand{\vDel}{v_\Delta}
\newcommand{\delv}{\delta v}
\newcommand{\uDel}{u_\Delta}
\newcommand{\delu}{\delta u}





